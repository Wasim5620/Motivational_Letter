%%%%%%%%%%%%%%%%%%%%%%%%%%%%%%%%%%%%%%%%%
% Medium Length Graduate Curriculum Vitae
% LaTeX Template
% Version 1.1 (9/12/12)
%
% This template has been downloaded from:
% http://www.LaTeXTemplates.com
%
% Original author:
% Rensselaer Polytechnic Institute (http://www.rpi.edu/dept/arc/training/latex/resumes/)
%
% Important note:
% This template requires the res.cls file to be in the same directory as the
% .tex file. The res.cls file provides the resume style used for structuring the
% document.
%
%%%%%%%%%%%%%%%%%%%%%%%%%%%%%%%%%%%%%%%%%

%----------------------------------------------------------------------------------------
%	PACKAGES AND OTHER DOCUMENT CONFIGURATIONS
%----------------------------------------------------------------------------------------

\documentclass[]{res} % Use the res.cls style, the font size can be changed to 11pt or 12pt here

\usepackage{helvet} % Default font is the helvetica postscript font
%\usepackage{newcent} % To change the default font to the new century schoolbook postscript font uncomment this line and comment the one above

\setlength{\textwidth}{5.4in} % Text width of the document

\begin{document}

%----------------------------------------------------------------------------------------
%	NAME AND ADDRESS SECTION
%----------------------------------------------------------------------------------------

\moveleft.15\hoffset\centerline{\large\bf Olzhas Shaikenov} % Your name at the top
\moveleft\hoffset\vbox{\hrule width\resumewidth height 1pt}\smallskip % Horizontal line after name; adjust line thickness by changing the '1pt'
\moveleft.15\hoffset\centerline{olzhas.shaikenov@nu.edu.kz} 
\moveleft.15\hoffset\centerline{}
\moveleft.15\hoffset\centerline{}

%----------------------------------------------------------------------------------------
\moveleft.20\hoffset\centerline{Motivational letter}

At the 3rd year of my undergraduate studies I understood that I really need to select the electrical engineering path which I will choose at my master degree or maybe even further. For about a year I was enrolled in Aresh Dadlani's research which mainly focuses on epidemic modelling. We have done a lot of mathematical analysis and simulations, but all the work that we have performed was only theoretical without significant practical usage.

So, this year I was worried am I really interested in such a theoretical work and what should I do. I did not even think about circuit design since the previous courses about MOSFET circuits were not actually well thought and useful.

Analog Circuit Design lectures have really interested me since the beginning of the semester because I finally started to understand what is MOSFET and how it works in terms of input voltages, current equation, and, generally, what is going on. Basically, we should already know this concepts, but it did not happen and this information was new for me.

After yesterday's lecture I think that I'm really interested in RF circuits and topics that were discussed at this stage. Further, I will study these topics more deeply and try to really understand the fundamental ideas and concepts which will help in the research. Hope for a productive and interesting work.


\begin{resume}

%----------------------------------------------------------------------------------------
%	OBJECTIVE SECTION
%----------------------------------------------------------------------------------------
 

 


\end{resume}
\end{document}